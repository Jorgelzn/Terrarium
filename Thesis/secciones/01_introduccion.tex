\chapter{Introduction}
%%---------------------------------------------------------



\section{Motivation}

The field of artificial intelligence has been increasing it´s popularity in the last years. Since the publication of the paper "Attention is all you need" \cite{attention}, the transformer arquitecture that it introduced has brought the creation of gigantic models cappable of processsing huge amounts of data. This is the case for the large language models and the generative ones.

However all these improvements are based in the classic core idea of suppervised learning, with the only difference been the arquitectures that allow much more examples in the training process. With this scenario in mind, it is interesting that we take a step back and consider exploring other algorithms and learning methods that do not only deppend on raw data and computation power.

Reinforcement Learning tries to model behaviour for intelligent agents with no more data than the input that the proper agents receive from the environment. This family of algorithms are heavily based in game Theory and the concept of reward. As stated in the publication "Reward is Enough"  \cite{rewardIsEnough}, the idea of receiving possitive feedback creates a strong guidance for learning complex tasks.

\section{Objectives}

\section{Problem Description}

\section{State of the art}

%%---------------------------------------------------------